% !TeX spellcheck = sr_LATN-SerbianLatin
\documentclass[a4paper,12pt]{report}
\usepackage[T1,OT1]{fontenc}
\usepackage[utf8]{inputenc}
\usepackage[serbian]{babel}
\usepackage{graphicx}
\usepackage{lmodern}
\usepackage[a4paper, portrait, margin=1in]{geometry}
\setcounter{secnumdepth}{3}
\setcounter{tocdepth}{4}

\title{\Large Elektrotehnički fakultet u Beogradu \\ 13S113PSI Principi Softverskog Inženjerstva}
\author{\Huge \textbf{Usluga na dlanu}\\ \ \\\includegraphics[scale=0.25]{Logo.png}\\}
\date{\Large Korišćeni alati pri implementaciji \\ \ \\ \ \\ \Large   Tim: Under pressure \\ Filip Janjić 2019/0116 \\ Jana Pašajlić 2019/0132 \\ Lazar Premović 2019/0091  \\ \  \\ \  \\\large Verzija 1.0}
\renewcommand{\contentsname}{Sadržaj}
\renewcommand{\thesection}{\arabic{section}}

\begin{document}

\maketitle

\begin{center}
\section*{Istorija izmena}
\begin{tabular}{ |l|c|p{5.5cm}|l|}
\hline
\textbf{Datum} & \textbf{Verzija} & \textbf{Kratak opis} & \textbf{Autor} \\ 
\hline
 01.06.2022. & 1.0  & Inicijalna verzija & Filip Janjić \\
\hline
 &  &  &  \\
 \hline
 &  &  &  \\
 \hline
\end{tabular}
\end{center}
\newpage
\date{\Huge \textbf{Korišćeni alati}}\\ \\ \\ 
Pri implementaciji aplikacije \textbf{Usluga na dlanu} korišćeni su sledeći alati: \\ 

\section{Frontend}

\begin{center}
    \title \textbf{HTML, CSS, JavaScript} \\
    \includegraphics[scale=0.2]{HtmlCssJs.png} \\
    Standardne tehnologije za dizajn, formatiranje i ponašanje veb-stranica
\end{center}

\begin{center}
    \title \textbf{Bootstrap 5} \\
    \includegraphics[scale=0.2]{Bootstrap.png} \\
    Radni okvir za formatiranje i dizajn odzivnih HTML stranica
\end{center}

\begin{center}
    \title \textbf{JQuery 1.10.2} \\
    \includegraphics[scale=0.4]{jquery.png} \\
    Biblioteka za kontrolu i upotrebu HTML DOM elemenata, CSS animaciju i AJAX
\end{center}

\begin{center}
    \title \textbf{jquery-locationpicker-plugin} \\
    \includegraphics[scale=1.1]{jquery-plugin.jpg} \\
    Proširenje JQuery biblioteke koje omogućava rad sa lokacijama, mapama i Google Maps API-jem
\end{center}

\begin{center}
    \title \textbf{Google Maps API} \\
    \includegraphics[scale=0.15]{google.png} \\
    Aplikativni interfejs kompanije Google koji povezuje našu veb-aplikaciju sa Google Maps sistemom i omogućava korišćenje i računanje lokacija
\end{center}

\begin{center}
    \title \textbf{AJAX} \\
    \includegraphics[scale=0.3]{ajax.png} \\
    Skup tehnika koje se koriste za spajanje klijentske strane sa serverskom putem asinhronih XML zahteva
\end{center}

\section{Backend}

\begin{center}
    \title \textbf{CodeIgniter 4.1.9} \\
    \includegraphics[scale=0.35]{codeigniter.png} \\
    Radni okvir sa MVC modelom korišćen kao osnova i podloga za celokupan projekat
\end{center}

\begin{center}
    \title \textbf{PHP 7.4.26} \\
    \includegraphics[scale=0.07]{PHP-logo.png} \\
    Programski jezik većinski korišćen za samu implementraciju
\end{center}

\begin{center}
    \title \textbf{phpmyadmin 5.1.1} \\
    \includegraphics[scale=0.15]{PhpMyAdmin.png} \\
    Administrativni alat za konekciju sa MySQL bazom
\end{center}

\begin{center}
    \title \textbf{MySQL} \\
    \includegraphics[scale=0.6]{mysql-logo.png} \\
    Sistem za upravljanje relacionim bazama podataka u kojem je modelovana potrebna baza
\end{center}

\begin{center}
    \title \textbf{WampServer64 i XAMPP} \\
    \includegraphics[scale=0.09]{wamp.png} 
    \includegraphics[scale=0.45]{xampp.png} \\
    Paketi koji u sebi sadrže servere koji su služili za razvoj aplikacije
\end{center}

\section{Opšte}

\begin{center}
    \title \textbf{Git, GitHub, Git Bash} \\
    \includegraphics[scale=0.45]{gitt.png} \includegraphics[scale=0.045]{GitHub.png} \includegraphics[scale=0.45]{git.png} \\ 
    Sistem za verzionisanje koda i fajlova (Git), remote repozitorijum host za isti (GitHub) i klijent (Git Bash)
\end{center}

\begin{center}
    \title \textbf{Microsoft Teams} \\
    \includegraphics[scale=0.06]{teams.png} \\
    Platforma za komunikaciju na kojoj su se održavali video sastanci i razmena bitnih fajlova i podsetnika
\end{center}

\begin{center}
    \title \textbf{Discord} \\
    \includegraphics[scale=0.2]{discord.png} \\
    Platforma za komunikaciju na kojoj je bila ostvarena kratka i hitna komunikacija, otvarana bitna pitanja i zakazivani sastanci
\end{center}

\begin{center}
    \title \textbf{OverLeaf, LaTeX} \\
    \includegraphics[scale=0.15]{overleaf.png}
    \includegraphics[scale=0.25]{latex-logo.png} \\
    Online alat za pisanje i formatiranje teksta u LaTeX-u, korišćenog za izradu sve dokumentacije
\end{center}


\begin{center}
    \title \textbf{Visual Studio Code} \\
    \includegraphics[scale=0.08]{vscode.png} \\
    Alat za pisanje, verzionisanje i formatiranje koda, sa dodatnim, naknadno instaliranim ekstenzijama (Intelephense, PHP formatter)
\end{center}

\end{document}