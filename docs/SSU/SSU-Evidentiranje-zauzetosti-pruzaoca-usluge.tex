\documentclass[a4paper,12pt]{report}
\usepackage[T1,OT1]{fontenc}
\usepackage[utf8]{inputenc}
\usepackage[serbian]{babel}
\usepackage{graphicx}
\usepackage{lmodern}
\usepackage{enumitem}
\usepackage{tabto} 
\usepackage[a4paper, portrait, margin=1in]{geometry}
\setcounter{secnumdepth}{3}
\setcounter{tocdepth}{4}
\renewcommand{\thesection}{\arabic{section}}
\renewcommand{\labelenumii}{\arabic{enumii}.}
\renewcommand{\labelenumiii}{\arabic{enumi}.\arabic{enumii}.\arabic{enumiii}}
\renewcommand{\labelenumiv}{\arabic{enumi}.\arabic{enumii}.\arabic{enumiii}.\arabic{enumiv}}

%EDIT HERE
\newcommand{\genitivfunkcionalnosti}{evidentiranja zauzetosti pružaoca usluge}
\newcommand{\dativfunkcionalnosti}{evidentiranju zauzetosti pružaoca usluge}
\newcommand{\inicijalniautor}{Jana Pašajlić}
\newcommand{\inicijalnidatum}{18.03.2022.}

\title{\Large Elektrotehnički fakultet u Beogradu \\ 13S113PSI Principi Softverskog Inženjerstva}
\author{\Huge Usluga na dlanu\\ \ \\ \ \\ \ \\ \ \\
	\Large \textbf{Specifikacija scenarija upotrebe funkcionalnosti}\\\Large \textbf{\genitivfunkcionalnosti} \\ \ \\}
\date{\Large   Tim: Under pressure \\ Filip Janjić 2019/0116 \\ Jana Pašajlić 2019/0132 \\ Lazar Premović 2019/0091  \\ \  \\ \  \\\large Verzija 1.0}
\renewcommand{\contentsname}{Sadržaj}
\renewcommand{\thesection}{\arabic{section}}

\begin{document}

\maketitle

\begin{center}
\section*{Istorija izmena}
\begin{tabular}{ |c|c|c|c| }
\hline
\textbf{Datum} & \textbf{Verzija} & \textbf{Kratak opis} & \textbf{Autor} \\ 
\hline
 \inicijalnidatum & 1.0  & Inicijalna verzija & \inicijalniautor \\
 \hline
 &  & &  \\
 \hline
 &  &  &  \\
 \hline
 &  &  &  \\
 \hline
\end{tabular}
\end{center}
\newpage

\tableofcontents

\newpage
\section{Uvod}
\subsection{Rezime}
    Definisanje scenarija slučaja upotrebe \genitivfunkcionalnosti, sa primerima odgovarajućih HTML stranica.
\subsection{Namena dokumenta i ciljne grupe}
    Dokument će koristiti svi članovi projektnog tima u razvoju projekta i testiranju, a može se koristiti i pri pisanju dokumentacije namenjene administratoru.
\subsection{Reference}
 \begin{enumerate}
    \item Projektni zadatak
    \item Uputstvo za pisanje specifikacije scenarija upotrebe funkcionalnosti 
    \item Guidelines - Use Case, Rational Unified Process 2000
    \item Guidelines - Use Case Storyboard, Rational Unified Process 2000
 \end{enumerate}
\subsection{Otvorena pitanja}
    \begin{center}
    \begin{tabular}{ |c|p{10cm}|l| }
    \hline
    \textbf{Redni broj} & \hspace{4cm} \textbf{Opis} & \textbf{Rešenje} \\ 
    \hline
     & & \\
     \hline
      & &   \\
     \hline
     & &   \\
     \hline
    \end{tabular}
    \end{center}
\section{Scenario \dativfunkcionalnosti}
\subsection{Kratak opis}
    Pružalac usluge može označiti određene vremenske periode kao slobodne ili zazete.Takođe, prilikom kontaktiranja pružaoca usluge od strane korisnika postoji specificirani vremenski period i u kalendaru pružaoca usluge se isti označava kao potencijalno zauzet. Potom pružalac usluge može da potvrdi ili da odbije taj termin, označavajući ga slobodnim ili zauzetim.

\subsection{Tok događaja}
    \paragraph*{Osnovni uspešni scenario za neoznačene zahteve pružanja usluge}
    \begin{enumerate}
        \item Sistem prikazuje pružaocu usluge da postoje zahtevi o pružanju usluga.
        \item Pružalac usluge otvara svoju komandnu tablu klikom na njenu ikonu.
        \item Sistem prikazuje zahteve o uslugama koje pružalac može da prihvati ili da odbije.
        \item Pružalac usluge otvara željeni zahtev.
        \item Razrešenje zahteva.
        \item Pružalac usluge se vraća na korak 4 ili prelazi na korak 7.
        \item Pružalac usluge završava pregled neoznačenih zahteva.
    \end{enumerate}
    
    \paragraph*{Proširenja uspešnog scenarija za neoznačene zahteve pružanja usluge} 
    \begin{enumerate}
        \item[5.1] Pružalac usluge želi da pruži uslugu.
        \begin{enumerate}[noitemsep,topsep=-8pt]
            \item U slučaju da pružalac usluge nije slobodan u vremenskom periodu predviđenim zahtevom, sistem onemogućava dugme za prihvatanje zahteva i pružalac se vraća u korak 6 osnovnog toka. 
            \item Pružalac usluge klikom na odgovarajuće dugme prihvata pružanje usluge.
            \item Kada korisnik koji je kreirao zahtev prihvati odobren zahtev, sistem ažurira kalendar pružaoca usluge i označava da je u navedenom vremenskom periodu pružalac zauzet.
            \item Zahtev o pružanju usluge se evidentira kao prihvaćen i ostaje vidljiv kod pružaoca usluge.
        \end{enumerate}
    
        \item[5.2] Pružalac usluge ne želi da pruži uslugu.
        \begin{enumerate}[noitemsep,topsep=-8pt]
            \item Pružalac usluge klikom na odgovarajuće dugme odbija pružanje usluge.
            \item Zahtev o pružanju usluge se evidentira kao neostvaren i zahtev se briše.
        \end{enumerate}
    \end{enumerate}
    
    \paragraph*{Scenario označavanja već prihvaćenih zahteva o uslugama kao neostvarene}
    \begin{enumerate}
        \item Pružalac usluge otvara svoju komandnu tablu klikom na njenu ikonu.
        \item Sistem prikazuje sekcije o prihvaćenim zahtevima.
        \item Pružalac usluge otvara sekciju prihvaćenih zahteva.
        \item Pružalac usluge bira zahtev za pružanje usluge.
        \item Pružalac usluge klikom na odgovarajuće dugme označava zahtev kao neostvaren.
        \item Sistem ažurira kalendar pružaoca usluge i označava da je u navedenom vremenskom periodu pružalac slobodan.
        \item Zahtev o pružanju usluge se evidentira kao neostvaren.
        \item Pružalac se vraća na korak 4 ili prelazi na korak 9.
        \item Pružalac usluge završava pregled prihvaćenih zahteva.
    \end{enumerate}
    
    
\subsection{Posebni zahtevi}
    Obratiti pažnju prilikom testiranja rada sistema na slučajeve istovremeno pristiglih zahteva, kako ne bi došlo do neželjenih preklapanja termina prilikom prihvatanja i odbijanja zahteva.
\subsection{Preduslovi}
    Potrebno je da se pružalac usluge prijavi na sistem. 
\subsection{Posledice}
    Nakon uspešno izvršene sekvence koraka, u bazi se ažuriraju informacije o zauzetosti pružaoca usluge, kao i u njegovom kalendaru .
\end{document}