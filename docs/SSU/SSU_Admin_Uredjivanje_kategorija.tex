% !TeX spellcheck = sr_LATN-SerbianLatin
\documentclass[a4paper,12pt]{report}
\usepackage[T1,OT1]{fontenc}
\usepackage[utf8]{inputenc}
\usepackage[serbian]{babel}
\usepackage{graphicx}
\usepackage{lmodern}
\usepackage[a4paper, portrait, margin=1in]{geometry}
\usepackage{enumitem}
\setcounter{secnumdepth}{3}
\setcounter{tocdepth}{3}

%EDIT HERE
\newcommand{\genitivfunkcionalnosti}{Administracije - uređivanja kategorija usluga}
\newcommand{\dativfunkcionalnosti }{Administraciji - uređivanju kategorija usluga}
\newcommand{\inicijalniautor}{Filip Janjić}
\newcommand{\inicijalnidatum}{20.03.2022.}

\title{\Large Elektrotehnički fakultet u Beogradu \\ 13S113PSI Principi Softverskog Inženjerstva}
\author{\Huge Usluga na dlanu\\ \ \\ \ \\ \ \\ \ \\
	\Large \textbf{Specifikacija scenarija upotrebe funkcionalnosti}\\\Large \textbf{\genitivfunkcionalnosti} \\ \ \\}
\date{\Large   Tim: Under pressure \\ Filip Janjić 2019/0116 \\ Jana Pašajlić 2019/0132 \\ Lazar Premović 2019/0091  \\ \  \\ \  \\
	\large Verzija 1.0}
\renewcommand{\contentsname}{Sadržaj}
\renewcommand{\thesection}{\arabic{section}}
\renewcommand{\labelenumii}{\arabic{enumii}.}
\renewcommand{\labelenumiii}{\arabic{enumi}.\arabic{enumii}.\arabic{enumiii}}
\renewcommand{\labelenumiv}{\arabic{enumi}.\arabic{enumii}.\arabic{enumiii}.\arabic{enumiv}}

\begin{document}
	
	\maketitle
	
	\begin{center}
		\section*{Istorija izmena}
			\begin{tabular}{ |l|l|l|l| }
				\hline
				\textbf{Datum} & \textbf{Verzija} & \textbf{Kratak opis} & \textbf{Autor} \\ 
				\hline
				\inicijalnidatum & 1.0  & Inicijalna verzija & \inicijalniautor \\
				\hline
				&  & &  \\
				\hline
				&  &  &  \\
				\hline
				&  &  &  \\
				\hline
			\end{tabular}
	\end{center}
	
	\newpage
	
	\tableofcontents
	
	\newpage
	
	\section{Uvod}
		\subsection{Rezime}
			Definisanje scenarija upotrebe pri \dativfunkcionalnosti, sa primerima odgovarajućih HTML stranica.
		\subsection{Namena dokumenta i ciljne grupe}
			Dokument će koristiti svi članovi projektnog tima u razvoju projekta i testiranju, a može se koristiti i pri pisanju dokumentacije namenjene administratoru.
		\subsection{Reference}
			\begin{enumerate}
				\item Projektni zadatak
				\item Uputstvo za pisanje specifikacije scenarija upotrebe funkcionalnosti
				\item Guidelines - Use Case, Rational Unified Process 2000
				\item Guidelines - Use Case Storyboard, Rational Unified Process 2000
			\end{enumerate}
		\subsection{Otvorena pitanja}
			\begin{tabular}{ |c|p{10cm}|l| }
				\hline
				\textbf{Redni broj} & \textbf{Opis} & \textbf{Rešenje} \\ 
				\hline
			    & &  \\
				\hline
			    &  & \\
				\hline
				&  & \\
				\hline
				&  & \\
				\hline
			\end{tabular}
	\section{Scenario \genitivfunkcionalnosti}
		\subsection{Kratak opis}
			Korisnik sa privijegijama administratora može uređivati kategorije usluga.
		\newpage
		\subsection{Tok događaja}
			\paragraph*{Osnovni uspešni scenario}
				\begin{enumerate}
					\item Admin bira opciju UREDI KATEGORIJE
					\item Admin bira DODAJ KATEGORIJU
					\item Unosi se ime nove kategorije i pritiska dugme DODAJ
					\item U bazu se upisuje nova kategorija i ista se pojavljuje na početnoj stranici korisnika.
				\end{enumerate}
			\paragraph*{Proširenja}
				\begin{enumerate}
					\item[2.1] Admin bira opciju UKLONI KATEGORIJU
						\begin{enumerate}[noitemsep,topsep=-8pt]
							\item Iz baze se briše odgovarajuća kategorija.
							\item Svi korisnici koji su pružaoci usluga date kategorije gube privilegije pružaoca za datu kategoriju.
						\end{enumerate}
				\end{enumerate}
		\subsection{Posebni zahtevi}
			Posebnu pažnju obratiti na efikasnost uklanjanja kategorije pri implementaciji, jer predstavlja vrlo vremenski zahtevnu operaciju.
		\subsection{Preduslovi}
			Administrator mora da bude prijavljen na sistem da bi mu se omogućila ova opcija.
		\subsection{Posledice}
			Nova kategorija se dodaje u bazu i omogućava prijavljivanje pružalaca.
			Kategorija može da bude izbačena iz baze gde svi pružaoci usluga u istoj gube privilegije pružaoca za tu kategoriju.
\end{document}