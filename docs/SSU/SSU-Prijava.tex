% !TeX spellcheck = sr_LATN-SerbianLatin
\documentclass[a4paper,12pt]{report}
\usepackage[T1,OT1]{fontenc}
\usepackage[utf8]{inputenc}
\usepackage[serbian]{babel}
\usepackage{graphicx}
\usepackage{lmodern}
\usepackage[a4paper, portrait, margin=1in]{geometry}
\usepackage{enumitem}
\setcounter{secnumdepth}{3}
\setcounter{tocdepth}{3}

%EDIT HERE
\newcommand{\genitivfunkcionalnosti}{prijave korisnika}
\newcommand{\dativfunkcionalnosti }{prijavi korisnika}
\newcommand{\inicijalniautor}{Lazar Premović}
\newcommand{\inicijalnidatum}{17.03.2022.}

\title{\Large Elektrotehnički fakultet u Beogradu \\ 13S113PSI Principi Softverskog Inženjerstva}
\author{\Huge Usluga na dlanu\\ \ \\ \ \\ \ \\ \ \\
	\Large \textbf{Specifikacija scenarija upotrebe funkcionalnosti}\\\Large \textbf{\genitivfunkcionalnosti} \\ \ \\}
\date{\Large   Tim: Under pressure \\ Filip Janjić 2019/0116 \\ Jana Pašajlić 2019/0132 \\ Lazar Premović 2019/0091  \\ \  \\ \  \\
	\large Verzija 1.1}
\renewcommand{\contentsname}{Sadržaj}
\renewcommand{\thesection}{\arabic{section}}
\renewcommand{\labelenumii}{\arabic{enumii}.}
\renewcommand{\labelenumiii}{\arabic{enumi}.\arabic{enumii}.\arabic{enumiii}}
\renewcommand{\labelenumiv}{\arabic{enumi}.\arabic{enumii}.\arabic{enumiii}.\arabic{enumiv}}

\begin{document}
	
	\maketitle
	
	\begin{center}
		\section*{Istorija izmena}
			\begin{tabular}{ |l|l|l|l| }
				\hline
				\textbf{Datum} & \textbf{Verzija} & \textbf{Kratak opis} & \textbf{Autor} \\ 
				\hline
				\inicijalnidatum & 1.0  & Inicijalna verzija & \inicijalniautor \\
				\hline
				12.04.2022. & 1.1 & Korekcije nakon formalne inspekcije & \inicijalniautor \\
				\hline
				&  &  &  \\
				\hline
				&  &  &  \\
				\hline
			\end{tabular}
	\end{center}
	
	\newpage
	
	\tableofcontents
	
	\newpage
	
	\section{Uvod}
		\subsection{Rezime}
			Definisanje scenarija upotrebe pri \dativfunkcionalnosti, sa primerima odgovarajućih html stranica.
		\subsection{Namena dokumenta i ciljne grupe}
			Dokument će koristiti svi članovi projektnog tima u razvoju projekta i testiranju, a može se koristiti i pri pisanju dokumentacije namenjene administratoru.
		\subsection{Reference}
			\begin{enumerate}
				\item Projektni zadatak
				\item Uputstvo za pisanje specifikacije scenarija upotrebe funkcionalnosti
				\item Guidelines - Use Case, Rational Unified Process 2000
				\item Guidelines - Use Case Storyboard, Rational Unified Process 2000
			\end{enumerate}
		\subsection{Otvorena pitanja}
			\begin{tabular}{ |c|p{10cm}|l| }
				\hline
				\textbf{Redni broj} & \textbf{Opis} & \textbf{Rešenje} \\ 
				\hline
			    1 & Da li treba onemogućiti prijavu nakon nekoliko neuspešnih pokušaja? &  \\
				\hline
			    &  & \\
				\hline
				&  & \\
				\hline
				&  & \\
				\hline
			\end{tabular}
	\section{Scenario \genitivfunkcionalnosti}
		\subsection{Kratak opis}
			Korisnik se prijavljuje korišćenjem korisničkog imena i šifre koje je uneo prilikom
			registracije. Nakon uspešne prijave korisniku se omogućava pregled kontakt informacija pružaoca usluge i
			kontaktiranje istog.
		\newpage
		\subsection{Tok događaja}
			\paragraph*{Osnovni uspešni scenario}
				\begin{enumerate}
					\item Korisnik otvara dijalog za prijavu klikom na dugme "Prijava"
					\item Korisnik unosi svoje korisničko ime
					\item Korisnik unosi svoju šifru
					\item Korisnik potvrđuje unete podatke klikom na dugme "Prijavi se"
					\item Sistem proverava da li korisnik sa datim korisničkim imenom postoji i da li je uneta pravilna šifra.
					\item Korisnik započinje sesiju vezanu za njegov korisnički nalog
				\end{enumerate}
			\paragraph*{Proširenja}
				\begin{enumerate}
					\item[6.1] Korisnik je uneo nepravilnu (ili praznu) vrednost za korisničko ime ili šifru
						\begin{enumerate}[noitemsep,topsep=-8pt]
							\item Podaci iz polja za šifru se brišu
							\item Ispisuje se poruka da je korisničko ime ili šifra neispravna
							\item Korisnik može opet pokušati sa prijavom od drugog koraka uspešnog scenarija
						\end{enumerate}
				\end{enumerate}
		\subsection{Posebni zahtevi}
			Šifre se u bazi podataka čuvaju heširane, te korisnički unos treba heširati pre poređenja sa podacima u bazi.
		\subsection{Preduslovi}
			Korisnik mora biti registrovan u sistemu.
		\subsection{Posledice}
			Ukoliko je prijava uspešna, korisnik započinje sesiju koja persistira do napuštanja sajta.
\end{document}