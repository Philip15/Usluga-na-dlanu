\documentclass[a4paper,12pt]{report}
\usepackage[T1,OT1]{fontenc}
\usepackage[utf8]{inputenc}
\usepackage[serbian]{babel}
\usepackage{graphicx}
\usepackage{lmodern}
\usepackage{enumitem}
\usepackage{tabto} 
\usepackage[a4paper, portrait, margin=1in]{geometry}
\setcounter{secnumdepth}{3}
\setcounter{tocdepth}{4}
\renewcommand{\thesection}{\arabic{section}}
\renewcommand{\labelenumii}{\arabic{enumii}.}
\renewcommand{\labelenumiii}{\arabic{enumi}.\arabic{enumii}.\arabic{enumiii}}
\renewcommand{\labelenumiv}{\arabic{enumi}.\arabic{enumii}.\arabic{enumiii}.\arabic{enumiv}}

%EDIT HERE
\newcommand{\genitivfunkcionalnosti}{uređivanja profila korisnika}
\newcommand{\dativfunkcionalnosti}{uređivanja profila korisnika}
\newcommand{\inicijalniautor}{Jana Pašajlić}
\newcommand{\inicijalnidatum}{18.03.2022.}

\title{\Large Elektrotehnički fakultet u Beogradu \\ 13S113PSI Principi Softverskog Inženjerstva}
\author{\Huge Usluga na dlanu\\ \ \\ \ \\ \ \\ \ \\
	\Large \textbf{Specifikacija scenarija upotrebe funkcionalnosti}\\\Large \textbf{\genitivfunkcionalnosti} \\ \ \\}
\date{\Large   Tim: Under pressure \\ Filip Janjić 2019/0116 \\ Jana Pašajlić 2019/0132 \\ Lazar Premović 2019/0091  \\ \  \\ \  \\\large Verzija 1.2}
\renewcommand{\contentsname}{Sadržaj}
\renewcommand{\thesection}{\arabic{section}}

\begin{document}

\maketitle

\begin{center}
\section*{Istorija izmena}
\begin{tabular}{ |c|c|c|c| }
\hline
\textbf{Datum} & \textbf{Verzija} & \textbf{Kratak opis} & \textbf{Autor} \\ 
\hline
 \inicijalnidatum & 1.0  & Inicijalna verzija & \inicijalniautor \\
 \hline
 24.03.2022. & 1.1  & Ispravljene slovne greške i kratak opis & \inicijalniautor  \\
 \hline
 12.04.2022. & 1.2  & Korekcije nakon formalne inspekcije & \inicijalniautor  \\
 \hline
 &  &  &  \\
 \hline
\end{tabular}
\end{center}
\newpage

\tableofcontents

\newpage
\section{Uvod}
\subsection{Rezime}
    Definisanje scenarija slučaja upotrebe izmene profila korisnika, sa primerima odgovarajućih HTML stranica.
\subsection{Namena dokumenta i ciljne grupe}
    Dokument će koristiti svi članovi projektnog tima u razvoju projekta i testiranju, a može se koristiti i pri pisanju dokumentacije namenjene administratoru.
\subsection{Reference}
 \begin{enumerate}
    \item Projektni zadatak
    \item Uputstvo za pisanje specifikacije scenarija upotrebe funkcionalnosti 
    \item Guidelines - Use Case, Rational Unified Process 2000
    \item Guidelines - Use Case Storyboard, Rational Unified Process 2000
 \end{enumerate}
\subsection{Otvorena pitanja}
    \begin{center}
    \begin{tabular}{ |c|p{10cm}|l| }
    \hline
    \textbf{Redni broj} & \hspace{4cm} \textbf{Opis} & \textbf{Rešenje} \\ 
    \hline
     1. & Da li dozvoliti izmenu email adrese korisnika? & \\
     \hline
      & &   \\
     \hline
     & &   \\
     \hline
    \end{tabular}
    \end{center}
\section{Scenario \dativfunkcionalnosti}
\subsection{Kratak opis}
    Nakon prijave na sistem korisnik ima privilegije izmene svog profila. Svaki korisnik može izmeniti ime, prezime, svoju profilnu sliku, korisničko ime i lozinku, dok pružalac usluga može dodatno menjati adresu poslovanja, kategoriju usluge koju pruža, kalendar zauzetosti, kao i opis na profilu. Posle izmene željenih informacija korisnik ostaje prijavljen na sistem i može nastaviti sa korišćenjem sajta.

\subsection{Tok događaja}
    \paragraph*{Osnovni uspešni scenario}
    \begin{enumerate}
        \item Korisnik se prijavljuje na sistem.
        \item Korisnik vrši pregled svog profila.
        \item Korisnik menja informacije i klikom na dugme Prijava ''Sačuvaj izmenu'' čuva izmene. 
        \item Korisnik nastavlja sa daljim korišćenjem sajta.
    \end{enumerate}
    
    \paragraph*{Proširenja} 
    \begin{enumerate}
        \item[2.1] Korisnik je odabrao izmenu korisničkog imena.
        \begin{enumerate}[noitemsep,topsep=-8pt]
            \item Korisnik unosi željeno korisničko ime.
            \item Sistem prikazuje da li korisničko ime već postoji.
            \item Ukoliko korisničko ime ne postoji u bazi, korisnik nastavlja sa korakom 2.
        \end{enumerate}
    
        \item[2.2] Korisnik je odabrao izmenu profilne sliku.
        \begin{enumerate}[noitemsep,topsep=-8pt]
             \item Korisnik postavlja željenu sliku i nastavlja sa korakom 2.
        \end{enumerate}
        
        \item[2.3] Korisnik (pružalac usluge) je odabrao izmenu adrese poslovanja.
        \begin{enumerate}[noitemsep,topsep=-8pt]
        \item Korisnik unosi novu adresu poslovanja i nastavlja sa korakom 2.
        \end{enumerate}
    
        \item[2.4] Korisnik (pružalac usluge) je odabrao izmenu kategorija usluga.
        \begin{enumerate}[noitemsep,topsep=-8pt]
            \item Korisnik bira kategorije usluga i nastavlja sa korakom 2.
        \end{enumerate}
        
        \item[2.5] Korisnik (pružalac usluge) je odabrao izmenu kalendara zauzetosti.
        \begin{enumerate}[noitemsep,topsep=-8pt]
            \item Korisnik označava željene vremenske periode kao zauzete ili slobodne i nastavlja sa korakom 2.
        \end{enumerate}
   
        \item[2.6] Korisnik (pružalac usluge) je odabrao izmenu opisa profila.
         \begin{enumerate}[noitemsep,topsep=-8pt]
        \item Koirsnik unosi nov opis profila i nastavlja sa korakom 2.
        \end{enumerate}
    \end{enumerate}
    
\subsection{Posebni zahtevi}
    Nema.
\subsection{Preduslovi}
    Potrebno je da se korisnik prijavi na sistem. 
\subsection{Posledice}
    Nakon uspešno izvršene sekvence koraka, u bazi se ažuriraju informacije o korisniku.
\end{document}