% !TeX spellcheck = sr_LATN-SerbianLatin
\documentclass[a4paper,12pt]{report}
\usepackage[T1,OT1]{fontenc}
\usepackage[utf8]{inputenc}
\usepackage[serbian]{babel}
\usepackage{graphicx}
\usepackage{lmodern}
\usepackage[a4paper, portrait, margin=1in]{geometry}
\usepackage{enumitem}
\setcounter{secnumdepth}{3}
\setcounter{tocdepth}{3}

\renewcommand{\thesection}{\arabic{section}}
\renewcommand{\labelenumii}{\arabic{enumii}.}
\renewcommand{\labelenumiii}{\arabic{enumi}.\arabic{enumii}.\arabic{enumiii}}
\renewcommand{\labelenumiv}{\arabic{enumi}.\arabic{enumii}.\arabic{enumiii}.\arabic{enumiv}}

%EDIT HERE
\newcommand{\genitivfunkcionalnosti}{registacije korisnika}
\newcommand{\dativfunkcionalnosti }{registraciji korisnika}
\newcommand{\inicijalniautor}{Jana Pašajlić}
\newcommand{\inicijalnidatum}{17.03.2022.}

\title{\Large Elektrotehnički fakultet u Beogradu \\ 13S113PSI Principi Softverskog Inženjerstva}
\author{\Huge Usluga na dlanu\\ \ \\ \ \\ \ \\ \ \\
	\Large \textbf{Specifikacija scenarija upotrebe funkcionalnosti}\\\Large \textbf{\genitivfunkcionalnosti} \\ \ \\}
\date{\Large   Tim: Under pressure \\ Filip Janjić 2019/0116 \\ Jana Pašajlić 2019/0132 \\ Lazar Premović 2019/0091  \\ \  \\ \  \\\large Verzija 1.0}
\renewcommand{\contentsname}{Sadržaj}
\renewcommand{\thesection}{\arabic{section}}

\begin{document}

\maketitle

\begin{center}
\section*{Istorija izmena}
\begin{tabular}{ |c|c|c|c| }
\hline
\textbf{Datum} & \textbf{Verzija} & \textbf{Kratak opis} & \textbf{Autor} \\ 
\hline
 \inicijalnidatum & 1.0  & Inicijalna verzija & \inicijalniautor \\
 \hline
 &  & &  \\
 \hline
 &  &  &  \\
 \hline
 &  &  &  \\
 \hline
\end{tabular}
\end{center}
\newpage

\tableofcontents

\newpage
\section{Uvod}
\subsection{Rezime}
Definisanje scenarija upotrebe pri \dativfunkcionalnosti, sa primerima odgovarajućih HTML stranica.
\subsection{Namena dokumenta i ciljne grupe}
Dokument će koristiti svi članovi projektnog tima u toku razvoja projekta i testiranja aplikacije. Može se koristiti prilikom izrade uputstva za upotrebu.
\subsection{Reference}
    \begin{enumerate}
        \item Projektni zadatak
        \item Uputstvo za pisanje specifikacije scenarija upotrebe funkcionalnosti 
        \item Guidelines - Use Case, Rational Unified Process 2000
        \item Guidelines - Use Case Storyboard, Rational Unified Process 2000
    \end{enumerate}
\subsection{Otvorena pitanja}
    \begin{center}
    \begin{tabular}{ |c|p{10cm}|l| }
    \hline
    \textbf{Redni broj} & \hspace{4cm} \textbf{Opis} & \textbf{Rešenje} \\ 
    \hline
         1. & Da li treba slati verifikacioni kod na email 
         korisnika koji se registruje? & \\
         \hline
          2.& Da li trajno pamtiti zahteve za kreiranje naloga? &  \\
         \hline
         & &   \\ 
         \hline
    \end{tabular}
    \end{center}
    
\section{Scenario \genitivfunkcionalnosti}
\subsection{Kratak opis}
    Korisnik se registruje unošenjem jedinstvenog korisničkog imena, jedinstvene email adrese i lozinke. Bez registracije i kreiranja naloga korisnik nije u mogućnosti da vidi kontakt informacije o pružaocima usluga. Ukoliko korisnik želi da bude pružalac usluge, potrebno je da unese dodatne podatke (kategoriju usluge i adresu poslovanja). Nakon korektno unetih podataka, na email adresu korisnika se šalje verifikacioni kod čijim se unošenjem prilikom registracije potvrđuje ispravnost unete email adrese, kreira se nalog i korisnik se registruje na sajt. 
    \newpage
\subsection{Tok događaja}
    \paragraph*{Osnovni uspešni scenario}
    \begin{enumerate}
        \item Korisnik unosi korisničko ime.
        \item Korisnik unosi email adresu.
        \item Korisnik unosi lozinku.
        \item Korisnik označava da li želi da bude pružalac usluge.
        \item Korisnik mora da označi saglasnost o uslovima korišćenja.
        \item Sistem šalje verifikacioni kod na unetu korisničku email adresu, čijim daljim unosom korisnik potvrđuje da je uneta korektna email adresa. Unete informacije o korisniku i se upisuje u bazu zahteva za kreiranje naloga. 
        \item Korisnik unosi verifikacioni kod koji mu je dostavljen na email adresu.
        \item Sistem unosi nalog korisnika u bazu.
        \item Nakon 24h zahtev za kreiranje naloga korisnika se briše iz baze.
    \end{enumerate}

    \paragraph*{Proširenja} 
    \begin{enumerate}
        \item[1.1] Korisnik je uneo već postojeće korisničko ime.
            \begin{enumerate}[noitemsep,topsep=-8pt]
                \item Sistem prikazuje da je korisničko ime zauzeto.
            \end{enumerate}
   
        \item[2.1] Korisnik je uneo email adresu koja postoji u bazi.
        \begin{enumerate}[noitemsep,topsep=-8pt]
            \item Sistem prikazuje da je email adresa zauzeto.
        \end{enumerate}
    
        \item[4.1] Korisnik je označio da želi da bude pružalac usluge.
            \begin{enumerate}[noitemsep,topsep=-8pt]
                \item Korisnik bira kategorije usluga.
                \item Korisnik unosi adresu poslovanja.
            \end{enumerate}
    
        \item[6.1] Korisnik nije uneo sve obavezne informacije.
            \begin{enumerate}[noitemsep,topsep=-8pt]
                \item Sistem prikazuje poruku da treba popuniti sva obavezna polja.
            \end{enumerate}
 
        \item[6.2] Korisnik nije validnu email adresu.
             \begin{enumerate}[noitemsep,topsep=-8pt]
                \item Sistem prikazuje poruku da uneta email adresa nije validna.
                \item Korisnik se vraća u korak 2.
            \end{enumerate}
    \end{enumerate}
    
\subsection{Posebni zahtevi}
    Korisnikova lozinka se u bazi čuva kao enkriptovana.
\subsection{Preduslovi}
    Potrebno je da korisnik poseduje email adresu. 
\subsection{Posledice}
    Nakon uspešno izvršene sekvence koraka, u bazu se trajno upisuje korsniko sa svojim informacijama i dalje može pristupati sajtu prijavom pomoću korisničkog imena i lozinke.
\end{document}