% !TeX spellcheck = sr_LATN-SerbianLatin
\documentclass[a4paper,12pt]{report}
\usepackage[T1,OT1]{fontenc}
\usepackage[utf8]{inputenc}
\usepackage[serbian]{babel}
\usepackage{graphicx}
\usepackage{lmodern}
\usepackage[a4paper, portrait, margin=1in]{geometry}
\usepackage{enumitem}
\setcounter{secnumdepth}{3}
\setcounter{tocdepth}{3}

%EDIT HERE
\newcommand{\genitivfunkcionalnosti}{ostavljanja recenzija i komentara}
\newcommand{\dativfunkcionalnosti }{ostavljanju recenzija i komentara}
\newcommand{\inicijalniautor}{Lazar Premović}
\newcommand{\inicijalnidatum}{18.03.2022.}

\title{\Large Elektrotehnički fakultet u Beogradu \\ 13S113PSI Principi Softverskog Inženjerstva}
\author{\Huge Usluga na dlanu\\ \ \\ \ \\ \ \\ \ \\
	\Large \textbf{Specifikacija scenarija upotrebe funkcionalnosti}\\\Large \textbf{\genitivfunkcionalnosti} \\ \ \\}
\date{\Large   Tim: Under pressure \\ Filip Janjić 2019/0116 \\ Jana Pašajlić 2019/0132 \\ Lazar Premović 2019/0091  \\ \  \\ \  \\
	\large Verzija 1.1}
\renewcommand{\contentsname}{Sadržaj}
\renewcommand{\thesection}{\arabic{section}}
\renewcommand{\labelenumii}{\arabic{enumii}.}
\renewcommand{\labelenumiii}{\arabic{enumi}.\arabic{enumii}.\arabic{enumiii}}
\renewcommand{\labelenumiv}{\arabic{enumi}.\arabic{enumii}.\arabic{enumiii}.\arabic{enumiv}}

\begin{document}
	
	\maketitle
	
	\begin{center}
		\section*{Istorija izmena}
			\begin{tabular}{ |l|l|l|l| }
				\hline
				\textbf{Datum} & \textbf{Verzija} & \textbf{Kratak opis} & \textbf{Autor} \\ 
				\hline
				\inicijalnidatum & 1.0  & Inicijalna verzija & \inicijalniautor \\
				\hline
				12.04.2022. & 1.1 & Korekcije nakon formalne inspekcije & \inicijalniautor \\
				\hline
				&  &  &  \\
				\hline
				&  &  &  \\
				\hline
			\end{tabular}
	\end{center}
	
	\newpage
	
	\tableofcontents
	
	\newpage
	
	\section{Uvod}
		\subsection{Rezime}
			Definisanje scenarija upotrebe pri \dativfunkcionalnosti, sa primerima odgovarajućih html stranica.
		\subsection{Namena dokumenta i ciljne grupe}
			Dokument će koristiti svi članovi projektnog tima u razvoju projekta i testiranju, a može se koristiti i pri pisanju dokumentacije namenjene administratoru.
		\subsection{Reference}
			\begin{enumerate}
				\item Projektni zadatak
				\item Uputstvo za pisanje specifikacije scenarija upotrebe funkcionalnosti
				\item Guidelines - Use Case, Rational Unified Process 2000
				\item Guidelines - Use Case Storyboard, Rational Unified Process 2000
			\end{enumerate}
		\subsection{Otvorena pitanja}
			\begin{tabular}{ |c|p{10cm}|l| }
				\hline
				\textbf{Redni broj} & \textbf{Opis} & \textbf{Rešenje} \\ 
				\hline
			    1 & Da li bi mogućnost ostavljanja recenzije trebalo da istekne nakon nekog vremena? &  \\
				\hline
			    &  & \\
				\hline
				&  & \\
				\hline
				&  & \\
				\hline
			\end{tabular}
	\section{Scenario \genitivfunkcionalnosti}
		\subsection{Kratak opis}
			Nakon ostvarivanja nekog zahteva, tj. pružanja neke usluge, korisnik može da ostavi brojčanu ocenu i opcioni tekstualni komentar na uslugu koja mu je pružena.
		\newpage
		\subsection{Tok događaja}
			\paragraph*{Osnovni uspešni scenario}
				\begin{enumerate}
					\item Sistem prikazuje korisniku da postoji usluga za koju može da ostavi recenziju
					\item Korisnik otvara svoju komandnu tablu klikom na njenu ikonu
					\item Sistem prikazuje usluge za koje korisnik može da ostavi recenziju
					\item Korisnik popunjava ocenu i opciono ostavlja komentar
					\item Korisnik potvrđuje recenziju klikom na odgovarajuće dugme
					\item Sistem beleži recenziju u bazi podataka i zahvaljuje se korisniku na odvojenom vremenu
				\end{enumerate}
			\paragraph*{Proširenja}
				\begin{enumerate}
					\item[4.1] Korisnik ne želi da ostavi recenziju
						\begin{enumerate}[noitemsep,topsep=-8pt]
							\item Korisnik uklanja uslugu sa spiska bez davanja recenzije klikom na odgovarajuće dugme
							\item Sistem označava recenziju kao odbijenu i uklanja je sa komandne table
						\end{enumerate}
					\item[4.4] Korisnik nije popunio ocenu
						\begin{enumerate}[noitemsep,topsep=-8pt]
							\item Sistem obaveštava korisnika da je ocena obavezno polje
							\item Korisnik može popuniti ocenu i nastaviti sa scenariom
						\end{enumerate}
				\end{enumerate}
		\subsection{Posebni zahtevi}
			Nema.
		\subsection{Preduslovi}
			Korisnik je prijavljen. Postoji usluga koja je realizovana i za koju korisnik još uvek nije ostavio recenziju.
		\subsection{Posledice}
			Ažurirana prosečna ocena i komentar se pojavljuju na stranici ponuđača.
\end{document}