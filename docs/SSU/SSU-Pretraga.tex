% !TeX spellcheck = sr_LATN-SerbianLatin
\documentclass[a4paper,12pt]{report}
\usepackage[T1,OT1]{fontenc}
\usepackage[utf8]{inputenc}
\usepackage[serbian]{babel}
\usepackage{graphicx}
\usepackage{lmodern}
\usepackage[a4paper, portrait, margin=1in]{geometry}
\usepackage{enumitem}
\setcounter{secnumdepth}{3}
\setcounter{tocdepth}{3}

%EDIT HERE
\newcommand{\genitivfunkcionalnosti}{pretrage i filtriranja usluga}
\newcommand{\dativfunkcionalnosti }{pretrazi i filtriranju usluga}
\newcommand{\inicijalniautor}{Lazar Premović}
\newcommand{\inicijalnidatum}{18.03.2022.}

\title{\Large Elektrotehnički fakultet u Beogradu \\ 13S113PSI Principi Softverskog Inženjerstva}
\author{\Huge Usluga na dlanu\\ \ \\ \ \\ \ \\ \ \\
	\Large \textbf{Specifikacija scenarija upotrebe funkcionalnosti}\\\Large \textbf{\genitivfunkcionalnosti} \\ \ \\}
\date{\Large   Tim: Under pressure \\ Filip Janjić 2019/0116 \\ Jana Pašajlić 2019/0132 \\ Lazar Premović 2019/0091  \\ \  \\ \  \\
	\large Verzija 1.1}
\renewcommand{\contentsname}{Sadržaj}
\renewcommand{\thesection}{\arabic{section}}
\renewcommand{\labelenumii}{\arabic{enumii}.}
\renewcommand{\labelenumiii}{\arabic{enumi}.\arabic{enumii}.\arabic{enumiii}}
\renewcommand{\labelenumiv}{\arabic{enumi}.\arabic{enumii}.\arabic{enumiii}.\arabic{enumiv}}

\begin{document}
	
	\maketitle
	
	\begin{center}
		\section*{Istorija izmena}
			\begin{tabular}{ |l|l|l|l| }
				\hline
				\textbf{Datum} & \textbf{Verzija} & \textbf{Kratak opis} & \textbf{Autor} \\ 
				\hline
				\inicijalnidatum & 1.0  & Inicijalna verzija & \inicijalniautor \\
				\hline
				12.04.2022. & 1.1 & Korekcije nakon formalne inspekcije & \inicijalniautor \\
				\hline
				&  &  &  \\
				\hline
				&  &  &  \\
				\hline
			\end{tabular}
	\end{center}
	
	\newpage
	
	\tableofcontents
	
	\newpage
	
	\section{Uvod}
		\subsection{Rezime}
			Definisanje scenarija upotrebe pri \dativfunkcionalnosti, sa primerima odgovarajućih html stranica.
		\subsection{Namena dokumenta i ciljne grupe}
			Dokument će koristiti svi članovi projektnog tima u razvoju projekta i testiranju, a može se koristiti i pri pisanju dokumentacije namenjene administratoru.
		\subsection{Reference}
			\begin{enumerate}
				\item Projektni zadatak
				\item Uputstvo za pisanje specifikacije scenarija upotrebe funkcionalnosti
				\item Guidelines - Use Case, Rational Unified Process 2000
				\item Guidelines - Use Case Storyboard, Rational Unified Process 2000
			\end{enumerate}
		\subsection{Otvorena pitanja}
			\begin{tabular}{ |c|p{10cm}|l| }
				\hline
				\textbf{Redni broj} & \textbf{Opis} & \textbf{Rešenje} \\ 
				\hline
			    1 & Da li želimo da dodamo i opcije za filtriranje po oceni i udaljenosti (pored sortiranja), npr. prikaži samo ponuđače sa ocenom većom od X ili bliže od Y? &  \\
				\hline
			    2 & Da li želimo da dodamo i opciju za sortiranje po najskorijem slobodnom terminu?  & \\
				\hline
				&  & \\
				\hline
				&  & \\
				\hline
			\end{tabular}
	\section{Scenario \genitivfunkcionalnosti}
		\subsection{Kratak opis}
			Pri pregledu ponuđača usluga korisnik moze suziti izbor ponuđača ili ga urediti na osnovu željenih kriterijuma.
			Kriterijumi za filtriranje su kategorija usluge i dostupnost ponuđača u određenom vremenskom periodu.
			Kriterijumi za sortiranje su ocena ponuđača i geografska udaljenost ponuđača od korisnički definisane lokacije.
		\newpage
		\subsection{Tok događaja}
			\paragraph*{Osnovni uspešni scenario}
				\begin{enumerate}
					\item Korisnik bira kriterijume za filtriranje i/ili sortiranje
					\item Sistem prikazuje potencijalne ponuđače koji ispunjavaju zadate kriterijume, sortirane po izabranom ili podrazumevanom kriterijumu. Za svakog ponuđača se prikazuje profilna slika, ime, kategorija, kratak opis i ocena.
					\item Korisnik odabira jednog od ponuđača klikom na njegovu karticu.
					\item Sistem Prikazuje detalje o izabranoj ponudi. Korisnik može nastaviti sa scenariom kontaktiranja pružaoca usluge ili kreiranja zahteva za uslugom.
				\end{enumerate}
			\paragraph*{Proširenja}
				\begin{enumerate}
					\item[1.1] Korisnik odabira jednu od kategorija usluga iz menija sa leve strane
						
					\item[1.2] Korisnik odabira period dostupnosti na osnovu kog želi da filtrira ponuđače
						\begin{enumerate}[noitemsep,topsep=-8pt]
							\item Korisnik omogućava ovu opciju klikom na odgovarajuće dugme u meniju na levoj strani
							\item Sistem prikazuje dodatne kontrole koje omogućavaju izbor i potvrdu vremenskog perioda
							\item Korisnik odabira početno i krajnje vreme perioda za koji želi da prikaže slobodne ponuđače
							\item Korisnik odabira početni i krajnji datum perioda za koji želi da prikaže slobodne ponuđače
							\item Korisnik potvrđuje odabran vremenski period klikom na odgovarajuće dugme
						\end{enumerate}
					\item[1.3] Korisnik odabira sortiranje na osnovu ocene koristeći padajući meni sa leve strane
						
					\item[1.4] Korisnik odabira sortiranje na osnovu geografske udaljenosti koristeći padajući meni sa leve strane
						\begin{enumerate}[noitemsep,topsep=-8pt]
							\item Sistem prikazuje mapu
							\item Korisnik bira tačku u odnosu na koju se računa udaljenost klikom na željenu lokaciju
						\end{enumerate}
					\item[2.1] Ne postoji nijedan ponuđač koji ispunjava sve uslove
						\begin{enumerate}[noitemsep,topsep=-8pt]
							\item Prikazuje se poruka da ne postoji ni jedan ponuđač koji ispunjava uslove i moli se korisnik da proširi kriterijum pretrage
						\end{enumerate}
				\end{enumerate}
		\subsection{Posebni zahtevi}
			Korisnik može izabrati bilo koju kombinaciju kriterijuma za pretragu ali samo jedan kriterijum za sortiranje.
			Podrazumevani kriterijum za pretragu prikazuje sve ponuđače bez obzira na kategoriju usluge i vreme dostupnosti.
			Podrazumevani kriterijum za sortiranje je sortiranje po ocenama.
		\subsection{Preduslovi}
			Nema.
		\subsection{Posledice}
			Korisnik prelazi na profil ponuđača odakle može nastaviti sa scenariom kontaktiranja pružaoca usluge ili kreiranja zahteva za uslugom.
\end{document}